\documentclass[12pt]{report}

\usepackage{times}
\usepackage{xcolor}
\usepackage{dirtree}
\usepackage{graphicx}
\usepackage{hyperref}
\usepackage{indentfirst}
\usepackage[margin=1in]{geometry}

\hypersetup{colorlinks=true, linkcolor=teal}
\setlength{\DTbaselineskip}{25pt}
\renewcommand{\DTstyle}{\rmfamily\large}

\newcommand{\n}{\par}
\newcommand{\br}{\vspace{1 em}\n}

\title{SBA Web Application Report}
\author{David W.}
\date{Last Revised \today}

\begin{document}
\maketitle

\textbf{Introduction}
\br
This report focuses on Assignee, a web application born in response to the SBA task:
Design a web application that implements an assignment system.
\br
This report contains the following chapters of different aspects:
\begin{itemize}
	\item \hyperref[overview]{Overview}:\\
	      Project objectives and structure;
	\item \hyperref[data-layer]{Data Layer}:\\
	      Relational database design and implementation;
	      %MO DEV unfinished
	\item \hyperref[application-layer]{Application Layer}:\\
	      Site back-end components design and implementation;
	\item \hyperref[presentation-layer]{Presentation Layer}:\\
	      Site front-end components design and implementation;
	\item \hyperref[security]{Security}:\\
	      User authentication and cross-layer interaction security;
	\item \hyperref[accessibility]{Accessibility}:\\
	      Site accessibility considerations and localization (l10n);
	\item \hyperref[performance]{Performance}:\\
	      Application performance optimizations;
	\item \hyperref[quality-assurance]{Quality Assurance}:\\
	      Application unit tests and integration tests;
	\item \hyperref[tools-used]{Tools Used}:\\
	      Applications, extensions, packages and libraries used during development;
\end{itemize}
\tableofcontents

\chapter{Overview} \label{overview}
\section{Project Objectives} \label{overview.project-objectives}
The next few pages is a brief overview of what could be done by users of different roles.\n
A few systems are decoupled to handle complicated application logic.
As of now we have 3 core systems: the User System, the Team System and the Assignment System,
as will be described below.
\br
Notes:\n
Only workflow related points are listed, GUI related points are not.\n
Only roughly describes each task (e.g. login),
the rest of the details (e.g. via username / with auth etc.) are given later in the report.
\br
Symbolic Notes:\n
...... $\rightarrow$ \{A\}: the user who performs the action now attains role A.\n
\{B\} $\rightarrow$ \{C\}: the action will make a user with role B to attain role C.\n
\{D\} $\in$ \{E\}: the role D inherits from role E, users with role D will be able to do anything role E is able to do.
\subsection{User System} \label{overview.project-objectives.user-system}
Roles: \{Site Visitors\}, \{Logged-in Users\}\n
\begin{itemize}
	\item \{Site Visitors\}
	      \begin{itemize}
		      \item Register for accounts;\null\hfill $\rightarrow$ \{Logged-in Users\}
		      \item Log-in to existing accounts;\null\hfill $\rightarrow$ \{Logged-in Users\}
	      \end{itemize}
	\item \{Logged-in Users\}
	      \begin{itemize}
		      \item Alter password;
		      \item Alter username or email;
		      \item Alter other settings;
		      \item Log-out of account;\null\hfill $\rightarrow$ \{Site Visitors\}
		      \item Delete the account;\null\hfill $\rightarrow$ \{Site Visitors\}
	      \end{itemize}
\end{itemize}\n
The User System is simply a casual user account system you will find in any web application,
where visitors are allowed to create accounts and ones with an account could log-in and manipulate the account.
\br
Authentication techniques are implemented to assist this system, which details are given in later chapters.
\br
The User System is the core of all systems possibly involved in this application,
including the \hyperref[overview.project-objectives.team-system]{Team System},
the \hyperref[overview.project-objectives.assignment-system]{Assignment System},
and any systems that might be added in the future (e.g. Posts and Comments System,
Instant Messaging System with P2P WebRTC or C/S WebSockets).
\subsection{Team System} \label{overview.project-objectives.team-system}
Roles: \{Users\}, \{Team Members\}, \{Team Monitors\}, \{Team Owners\}\n
\begin{itemize}
	\item \{Users\}
	      \begin{itemize}
		      \item Join teams;\null\hfill $\rightarrow$ \{Team Members\}
		      \item Create teams;\null\hfill $\rightarrow$ \{Team Owners\}
	      \end{itemize}
	\item \{Team Members\} $\in$ \{Users\}
	      \begin{itemize}
		      \item Alter per-team settings;
		      \item Leave the team;\null\hfill $\rightarrow$ \{Users\}
		      \item Comment on messages*;
	      \end{itemize}
	\item \{Team Monitors\} $\in$ \{Team Members\}
	      \begin{itemize}
		      \item Invite team members;\null\hfill \{Users\} $\rightarrow$ \{Team Members\}
		      \item Kick team members;\null\hfill \{Team Members\} $\rightarrow$ \{Users\}
		      \item Post messages*;
	      \end{itemize}
	\item \{Team Owners\} $\in$ \{Team Monitors\}
	      \begin{itemize}
		      \item Alter team name;
		      \item Alter team description;
		      \item Alter other team settings;
		      \item Appoint team monitors;\null\hfill \{Users\} $\rightarrow$ \{Team Monitors\}
		      \item Remove team monitors;\null\hfill \{Team Monitors\} $\rightarrow$ \{Users\}
		      \item Delete the team;\null\hfill $\rightarrow$ \{Users\}
	      \end{itemize}
\end{itemize}\n
Slightly isomorphic to Google Classroom,
users are allowed to create or join teams, and team owners could (de-)appoint team monitors to help manage the team.
Besides that, roles of different hierarchies could manage members and perform team actions
to different degrees and extents.
\br
The *post and comment features are not implemented, at least in the initial version of the web app,
as it is deemed to be of lower priority. It will be decoupled from this system if later implemented.
\br
The Team System is the core of the \hyperref[overview.project-objectives.assignment-system]{Assignment System}.
\subsection{Assignment System} \label{overview.project-objectives.assignment-system}
Roles: \{Team Owners\}, (\{Team Members\},) \{Assignees\}\n
\begin{itemize}
	\item \{Team Owners\}
	      \begin{itemize}
		      \item Author assignments;
		      \item Add attachments to assignments;
		      \item Assign assignments;\null\hfill \{Team Members\} $\rightarrow$ \{Assignees\}
		      \item De-assign assignments;\null\hfill \{Assignees\} $\rightarrow$ \{Team Members\}
		      \item Grade, comment and return submissions;
	      \end{itemize}
	\item \{Assignees\} $\in$ \{Team Members\}
	      \begin{itemize}
		      \item Submit assignments;
		      \item Add attachments to submissions;
	      \end{itemize}
\end{itemize}\n
The Assignment System is the core feature we are asked to complete.
The system works intuitively:\n
Team owners are allowed to assign assignments to some team members with a deadline.
The assignees are allowed to (draft and) hand in their submission,
after which the team owner could grade it, give comments and return it.
\br
An attachment system is designed to handle attachment files, to allow team owners/assignees
to attach more files to their assignments/submissions along the way
without the need to re-attach all prior files after quitting.
More details will be provided in later chapters.
\section{Project Structure} \label{overview.project-structure}
\subsection{Folder Structure} \label{overview.project-structure.folder-structure}
To keep a clear separation of concern and promote modularity,
the web application is separated into
the Data Layer, the Application Layer and the Presentation Layer.
\br
This could be seen in the project folder structure.
The folder structure is subject to frequent changes,
but the main shape will not differ vastly across revisions.
%MO REMINDER this dirtree shall be on a single page
\br
\dirtree{%
	.1 /.
	.2 design/.
	.3 database/.
	.3 report/.
	.2 site/.
	.3 backend/.
	.3 frontend/.
}
\vspace{2 em}
The directory design/ contains files that would not be served as a part of the actual web application site.
This includes this report, media assets used in this report etc.
\br
Notes:\n
The directory database/ corresponds to the Data Layer. It is kept under design/ since the hosted database
itself is independent of the site/ folder (only linked via database url in .env, details later),
so only the distribution .sql file that would be used to instantiate the database is kept.
Details are given in the chapter \hyperref[data-layer]{Data Layer}.
\br
The directory backend/ corresponds to the Application Layer.
Details are given in the chapter \hyperref[application-layer]{Application Layer}.
\br
The directory frontend/ corresponds to the Presentation Layer.
Details are given in the chapter \hyperref[presentation-layer]{Presentation Layer}.
\subsection{Repository} \label{overview.project-structure.repository}
The whole project folder has a repository and is available at
\href{https://github.com/CarbonicSoda/assignee}{GitHub/Carbonic\-Soda/assignee}
\br
With Git(Hub) we could review all past commits from the start of this project to the latest update or patch.
This not only eases version control and debugging,
but also allows us to work on back/front-end etc. simultaneously with branches.
\br
And of course the productivity benefits of having a repository do not stop here.

\chapter{Data Layer} \label{data-layer}

\section{Design}

% \chapter{Application Layer} \label{application-layer}

% \chapter{Presentation Layer} \label{presentation-layer}

% \chapter{Security} \label{security}

% \chapter{Accessibility} \label{accessibility}

% \chapter{Performance} \label{performance}

% \chapter{Quality Assurance} \label{quality-assurance}

% \chapter{Tools Used} \label{tools-used}

\end{document}
